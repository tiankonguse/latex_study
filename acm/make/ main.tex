\documentclass[a4paper]{article}

\usepackage{amsmath}
\usepackage{amssymb}
\usepackage{cmap}
\usepackage{geometry}
\usepackage{hyperref}
\usepackage{indentfirst}
\usepackage{xeCJK}
\usepackage{minted}

\geometry{margin=1in}

\setCJKmainfont[BoldFont={SimHei}]
{SimSun}
\setCJKmonofont{FangSong}

\newcommand{\cppcode}[1]{
    \inputminted[mathescape]{cpp}{source/#1}
}

\title{代码库}
\author{上海交通大学}
\date{\today}

\begin{document}

\maketitle

\tableofcontents

\clearpage

\section{计算几何}

\subsection{半平面交}

\subsubsection{$O(N^2)$}

\cppcode{half-plane-intersection-n2.cpp}

\subsubsection{$O(N \log N)$}

\cppcode{half-plane-intersection-nlogn.cpp}

\section{数据结构}

\subsection{$k$d树}

\cppcode{kd-tree.cpp}

\subsection{坚固的数据结构}

(Joke from \textbf{crazyb0y}) 

\subsubsection{坚固的线段树}

\cppcode{persistent-segment-tree.cpp}

\subsubsection{坚固的平衡树}

\cppcode{persistent-treap.cpp}

\subsection{后缀三姐妹}

\subsubsection{后缀数组}

\cppcode{suffix-array.cpp}

\subsubsection{后缀自动机}

\cppcode{suffix-automaton.cpp}

\subsection{最长回文串Manacher算法}

\texttt{palindrome[i]}是以$\frac{i}{2}$为对称中心的最长回文串长度

\cppcode{manacher.cpp}

\section{图论}

\section{数论}

\subsection{Millar-rabin}

\cppcode{millar-rabin.cpp}

\subsection{Polar Rho}

\cppcode{pollard-rho.cpp}

\subsection{快速傅里叶变换}

\cppcode{fast-fourier-transform.cpp}

\subsection{直线下格点统计}

计算$$\sum_{0 \leq i < n} \lfloor \frac{a + b \cdot i}{m} \rfloor$$
($n, m > 0, a, b \geq 0$)
\cppcode{lattice-count.cpp}

\section{Miscellaneous}

\subsection{环状最长公共子串}

\cppcode{cyclic-longest-common-string.cpp}

\subsection{二次剩余}

解方程$x^2 \equiv n \pmod p (p > 2)$, 
找$a$使得$\omega = a^2 - n$不是二次剩余,
则$$x \equiv (a + \sqrt{\omega})^{\frac{p + 1}{2}} \left(\bmod \frac{\mathbb{F}_p[x]}{(x^2 - \omega)}\right)$$

\subsection{五边形数定理}

设$p(n)$是$n$的拆分数,有$$p(n) = \sum_{k} (-1)^{k - 1} p\left(n - \frac{k(3k - 1)}{2}\right)$$

\subsection{球面三角公式}

设$a, b, c$是边长,$A, B, C$是所对的二面角,
有余弦定理$$\cos a = \cos b \cos c + \sin b \sin c \cos A$$
正弦定理$$\frac{\sin A}{\sin a} = \frac{\sin B}{\sin b} = \frac{\sin C}{\sin c}$$
三角形面积是$A + B + C - \pi$


\subsection{四面体体积公式}

$U, V, W, u, v, w$是四面体的$6$条棱,$U, V, W$构成三角形,$(U, u), (V, v), (W, w)$互为对棱,
则$$V = \frac{\sqrt{(s - 2a)(s - 2b)(s - 2c)(s - 2d)}}{192 uvw}$$
其中$$\left\{\begin{array}{lll}
        a & = & \sqrt{xYZ}, \\
        b & = & \sqrt{yZX}, \\
        c & = & \sqrt{zXY}, \\
        d & = & \sqrt{xyz}, \\
        s & = & a + b + c + d, \\ 
        X & = & (w - U + v)(U + v + w), \\
        x & = & (U - v + w)(v - w + U), \\
        Y & = & (u - V + w)(V + w + u), \\
        y & = & (V - w + u)(w - u + V), \\
        Z & = & (v - W + u)(W + u + v), \\
        z & = & (W - u + v)(u - v + W)
    \end{array}\right.$$

\subsection{牛顿恒等式}

设$$\prod_{i = 1}^n{(x - x_i)} = a_n + a_{n - 1} x + \dots + a_1 x^{n - 1} + a_0 x^n$$
$$p_k = \sum_{i = 1}^n{x_i^k}$$
则$$a_0 p_k + a_1 p_{k - 1} + \cdots + a_{k - 1} p_1 + k a_k = 0$$

特别地,对于$$|\mathbf{A} - \lambda \mathbf{E}| = (-1)^n(a_n + a_{n - 1} \lambda + \cdots + a_1 \lambda^{n - 1} + a_0 \lambda^n)$$
有$$p_k = \mathrm{Tr}(\mathbf{A}^k)$$

\end{document}