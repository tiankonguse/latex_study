\documentclass[12pt]{article}

\usepackage{CJK}
\usepackage[margin=0in]{geometry}
%format of the algorithm 
\usepackage{algorithm}               

%format of the algorithm 
\usepackage{algorithmic}             

%multirow for format of table 
\usepackage{multirow}                
\usepackage{amsmath} 
\usepackage{xcolor} 
\usepackage{graphics} 
\usepackage{graphicx} 
\usepackage{epsfig} 


\begin{document}
\begin{CJK}{UTF8}{gbsn}

%算法的开始
\begin{algorithm}[htb]
%算法的标题 
\caption{ Framework of ensemble learning for our system.}             

%给算法一个标签,这样方便在文中对算法的引用 
\label{alg:Framwork}                  

%不知[1]是干嘛的? 
\begin{algorithmic}[1]                

%算法的输入参数:Input 
\REQUIRE 描述\\                          
    The set of positive samples for current batch, $P_n$;\\
    The set of unlabelled samples for current batch, $U_n$;\\ 
    Ensemble of classifiers on former batches, $E_{n-1}$; 

\ENSURE 代码\\                           %算法的输出:Output 

    Ensemble of classifiers on the current batch,  $E_n$; 

\STATE Extracting the set of reliable negative and/or positive samples $T_n$ from $U_n$  with help of $P_n$; \label{code:fram:extract}      %算法的一个陈述,对应算法的一个步骤或公式之类的; \label{ code:fram:extract }对此行的标记,方便在文中引用算法的某个步骤 

\STATE Training ensemble of classifiers $E$ on $T_n \cup P_n$, with help of data in former batches; \label{code:fram:trainbase} 

\STATE $E_n=E_{n-1}\cup E$; \label{code:fram:add} 

\STATE Classifying samples in $U_n-T_n$ by $E_n$; \label{code:fram:classify} 

\STATE Deleting some weak classifiers in $E_n$ so as to keep the capacity of $E_n$; \label{code:fram:select} 

\RETURN $E_n$;                %算法的返回值 

\end{algorithmic} 

\end{algorithm} 



\begin{algorithm}[h] 

    \caption{An example for format For \& While Loop in Algorithm} 

    \begin{algorithmic}[1] 
    		\STATE for(int i=0;i<10;i++);
        \FOR{each $i\in [1,9]$} 
            \STATE initialize a tree $T_{i}$ with only a leaf (the root);\\ 
            \STATE $T=T\bigcup T_{i};$\\ 
        \ENDFOR 
        \FORALL {$c$ such that $c\in RecentMBatch(E_{n-1})$} \label{code:TrainBase:getc} 
            \STATE  $T=T \cup PosSample(c)$; \label{code:TrainBase:pos} 
        \ENDFOR; 

        \FOR{$i=1$; $i<n$; $i++$ } 
            \STATE $//$ Your source here; 
        \ENDFOR 

         

        \FOR{$i=1$ to $n$} 

            \STATE $//$ Your source here; 

        \ENDFOR 

         

        \STATE  $//$ Reusing recent base classifiers. \label{code:recentStart} 

        \WHILE {$(|E_n| \leq L_1 )and( D \neq \phi)$} 

            \STATE  Selecting the most recent classifier $c_i$ from $D$; 

            \STATE  $D=D-c_i$; 

            \STATE  $E_n=E_n+c_i$; 

        \ENDWHILE \label{code:recentEnd} 

         

    \end{algorithmic} 

\end{algorithm} 

\end{CJK}
\end{document}



